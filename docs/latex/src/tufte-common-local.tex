% !TEX encoding = UTF-8 Unicode

%%%%%%%%%%%%%%%%%%%%%%%%%%%%%%%%%%%%%%%%%%%%%%%%%%%%%%%%%%%%
% AMS Packages

\usepackage{amsmath, amsthm, amssymb, amsfonts}

%%%%%%%%%%%%%%%%%%%%%%%%%%%%%%%%%%%%%%%%%%%%%%%%%%%%%%%%%%%%
% Compiler Configuration Package

\usepackage{ifxetex}
%%%%%%%%%%%%%%%%%%%%%%%%%%%%%%%%%%%%%%%%%%%%%%%%%%%%%%%%%%%%
% Fonts Packages

\ifxetex
	 \usepackage[small,euler-digits,euler-hat-accent,OT1]{eulervm} %Load before mathspec and fontspec packages
	
	\usepackage{fontspec}
	
	
	\defaultfontfeatures{Mapping=tex-text,
						Ligatures={TeX,Common},
						Scale=MatchLowercase,
						Color=textcolor}

	\newfontfamily{\sectioningtitle}[
				 Extension      = .ttf,
				 UprightFont	= *,
				 BoldFont		= *-Bold,
			 	 ItalicFont		= *-Oblique,
			 	 BoldItalicFont = *-BoldOblique]
				 {Adobe Garamond Pro}
	
	\setmainfont[Extension      = .ttf,
				 UprightFont	= *,
				 BoldFont		= *-Bold,
			 	 ItalicFont		= *-Oblique,
			 	 BoldItalicFont = *-BoldOblique,
			 	 Numbers={OldStyle,Proportional}]
				 {Adobe Garamond Pro}
				 
	\setsansfont[Scale			= MatchUppercase,
    			 Extension		= .otf,
    			 UprightFont	= *-regular,
    			 BoldFont 		= *-bold,
    			 ItalicFont     = *-italic,
    			 BoldItalicFont = *-bolditalic]
    			 {texgyreheros}
				 
	%\setmonofont{Lucida Sans Typewriter}
	
	
     
\else
	\usepackage[utf8]{inputenc}
	\usepackage[small,euler-digits,euler-hat-accent,OT1]{eulervm} %
	\usepackage{mathpazo}
\fi
\usepackage{mathrsfs} 

\ifxetex
	% Begin of command necessary in Tufte to run with XeLaTeX
	\newcommand{\textls}[2][5]{%
    \begingroup\addfontfeatures{LetterSpace=#1}#2\endgroup
  }
  \renewcommand{\allcapsspacing}[1]{\textls[15]{#1}}
  \renewcommand{\smallcapsspacing}[1]{\textls[10]{#1}}
  \renewcommand{\allcaps}[1]{\textls[15]{\MakeTextUppercase{#1}}}
  \renewcommand{\smallcaps}[1]{\smallcapsspacing{\scshape\MakeTextLowercase{#1}}}
  \renewcommand{\textsc}[1]{\smallcapsspacing{\textsmallcaps{#1}}}
   % End of command necessary in Tufte to run with XeLaTeX
\fi

%%%%%%%%%%%%%%%%%%%%%%%%%%%%%%%%%%%%%%%%%%%%%%%%%%%%%%%%%%%%
% Language Packages

\ifxetex
	\usepackage{polyglossia}
	\setdefaultlanguage[variant=british]{english}
	%\setotherlanguages{french,portuges,italian}
	\setotherlanguages{portuges,italian} % French languages leads to serifed footnote font
\else
	\usepackage[english]{babel}
\fi

\usepackage{csquotes}


%%%%%%%%%%%%%%%%%%%%%%%%%%%%%%%%%%%%%%%%%%%%%%%%%%%%%%%%%%%%
% Graphics Packages

\usepackage{xcolor}
\usepackage{graphicx}
  % declare the path(s) where your graphic files are
  \graphicspath{{./imgs/}{./linear_regression/imgs/}{./classification/imgs/}{./trees/imgs/}{./hypothesis_testing/imgs/}}
  % and their extensions so you won't have to specify these with
  % every instance of \includegraphics
  \DeclareGraphicsExtensions{.pdf,.jpeg,.png}
  
\usepackage{epstopdf}
\usepackage{epsfig}
\usepackage{svg}

\definecolor{gblue900}{HTML}{0D47A1}
\definecolor{ggreen900}{HTML}{1B5E20}
\definecolor{gred900}{HTML}{B71C1C}




%%%%%%%%%%%%%%%%%%%%%%%%%%%%%%%%%%%%%%%%%%%%%%%%%%%%%%%%%%%%
% Hyperref Package
\usepackage{hyphenat}
\usepackage{hyperref}% backref linktocpage pagebackref
%
\hypersetup{
% Uncomment the line below to remove all links (to references, figures, tables, etc)
%draft, 
colorlinks=true, linktocpage=true, pdfstartpage=1, pdfstartview=FitV,
% Uncomment the line below if you want to have black links (e.g. for printing black and white)
%colorlinks=false, linktocpage=false, pdfborder={0 0 0}, pdfstartpage=3, pdfstartview=FitV, 
breaklinks=true, pdfpagemode=UseNone, pageanchor=true, pdfpagemode=UseOutlines,
plainpages=false, bookmarksnumbered, bookmarksopen=true, bookmarksopenlevel=1,
hypertexnames=true, pdfhighlight=/O, urlcolor=gred900, linkcolor=gblue900, citecolor=ggreen900,
%
% PDF file meta-information
%
pdftitle={Answers},
pdfauthor={Humberto Stein Shiromoto},
pdfsubject={},
pdfkeywords={},
pdfcreator={},
pdfproducer={LaTeX}
}

%%%%%%%%%%%%%%%%%%%%%%%%%%%%%%%%%%%%%%%%%%%%%%%%%%%%%%%%%%%%
% Sectioning customization
% add numbers to chapters, sections, subsections
\setcounter{secnumdepth}{2}
% to allow custom headings
\usepackage{titlesec}
%\usepackage{titling} % This package replaces Tufte's titlepage

% part format
\titleformat{\part}[display]
   {\relax\ifthenelse{\NOT\boolean{@tufte@symmetric}}{\begin{fullwidth}}{}}% format applied to label+text
   {\Huge\bfseries\sffamily Part\ \thepart}
   {0em}
   {\hrule\Huge\bfseries\sffamily}% before the title body
   [\ifthenelse{\NOT\boolean{@tufte@symmetric}}{\end{fullwidth}}{}]

% chapter format
\titleformat{\chapter}%
  [block]% shape
  {\relax\ifthenelse{\NOT\boolean{@tufte@symmetric}}{\begin{fullwidth}}{}}% format applied to label+text
  {\Huge\bfseries\sffamily\thechapter\ \raisebox{2pt}{|}\ }% label
  {0pt}% horizontal separation between label and title body
  {\Huge\bfseries\sffamily}% before the title body
  [\ifthenelse{\NOT\boolean{@tufte@symmetric}}{\end{fullwidth}}{}]
  
%  section format
\ifxetex
	\titleformat{\section}%
  	[hang]% shape
  	{\LARGE\bfseries\sectioningtitle}% format applied to label+text
  	{\LARGE\bfseries\sectioningtitle\thesection}% label
  	{5pt}% horizontal separation between label and title body
  	{}% before the title body
	[]% after the title body

% subsection format
	\titleformat{\subsection}%
  	{\Large\bfseries\sectioningtitle}% format applied to label+text
  	{\Large\bfseries\sectioningtitle\thesubsection}% label
  	{5pt}% horizontal separation between label and title body
  	{}% before the title body
  	[]% after the title body
\else
  \titleformat{\section}%
  	[hang]% shape
  	{\LARGE\bfseries}% format applied to label+text
  	{\LARGE\bfseries\thesection}% label
  	{5pt}% horizontal separation between label and title body
  	{}% before the title body
  	[]% after the title body

% subsection format
	\titleformat{\subsection}%
  	{\Large\bfseries}% format applied to label+text
  	{\Large\bfseries\thesubsection}% label
  	{5pt}% horizontal separation between label and title body
  	{}% before the title body
  	[]% after the title body
\fi
%%%%%%%%%%%%%%%%%%%%%%%%%%%%%%%%%%%%%%%%%%%%%%%%%%%%%%%%%%%%
% Useful Packages
\usepackage{imakeidx}         % allows index generation
\makeindex[columns=2, title=Alphabetical Index, intoc, options= -s index_style.ist]

\usepackage{mdframed}
\usepackage{todonotes}
\usepackage{enumerate}
\usepackage{paralist}
\usepackage{comment}
\usepackage{minitoc}
\dominitoc
\usepackage{rotating}
\usepackage{nicefrac}
\usepackage{cleveref}
\usepackage[caption=false]{subfig}
%%%%%%%%%%%%%%%%%%%%%%%%%%%%%%%%%%%%%%%%%%%%%%%%%%%%%%%%%%%%
% Listings Package

\usepackage{listings}
\definecolor{dkgreen}{rgb}{0,0.6,0}
\definecolor{gray}{rgb}{0.5,0.5,0.5}
\definecolor{mauve}{rgb}{0.58,0,0.82}
\lstset{ %
%  language=R,                     % the language of the code
  basicstyle=\footnotesize\ttfamily,       % the size of the fonts that are used for the code
  numbers=left,                   % where to put the line-numbers
  numberstyle=\tiny\color{gray},  % the style that is used for the line-numbers
  stepnumber=1,                   % the step between two line-numbers. If it's 1, each line
                                  % will be numbered
  numbersep=5pt,                  % how far the line-numbers are from the code
  backgroundcolor=\color{white},  % choose the background color. You must add \usepackage{color}
  showspaces=false,               % show spaces adding particular underscores
  showstringspaces=false,         % underline spaces within strings
  showtabs=false,                 % show tabs within strings adding particular underscores
  frame=single,                   % adds a frame around the code
  rulecolor=\color{black},        % if not set, the frame-color may be changed on line-breaks within not-black text (e.g. commens (green here))
  tabsize=2,                      % sets default tabsize to 2 spaces
  captionpos=b,                   % sets the caption-position to bottom
  breaklines=true,                % sets automatic line breaking
  breakatwhitespace=false,        % sets if automatic breaks should only happen at whitespace
  title=\lstname,                 % show the filename of files included with \lstinputlisting;
                                  % also try caption instead of title
  keywordstyle=\color{blue},      % keyword style
  commentstyle=\color{dkgreen},   % comment style
  stringstyle=\color{mauve},      % string literal style
  escapeinside={\%*}{*)},         % if you want to add a comment within your code
  morekeywords={*,...},            % if you want to add more keywords to the set
  resetmargins=False,
  xleftmargin=0pt,
  xrightmargin=0pt,
} 

%%%%%%%%%%%%%%%%%%%%%%%%%%%%%%%%%%%%%%%%%%%%%%%%%%%%%%%%%%%%
% Float Package
%\usepackage{float}
%\newfloat{lstfloat}{htbp}{lop}
%\floatname{lstfloat}{Listing}
%%%%%%%%%%%%%%%%%%%%%%%%%%%%%%%%%%%%%%%%%%%%%%%%%%%%%%%%%%%%
% Header and footnotes
%
\fancyhf{} % clear header and footer fields
\ifthenelse{\boolean{@tufte@twoside}}
  {\fancyhead[LE]{\sffamily\footnotesize{\thepage}\quad \raisebox{1pt}{|}\quad{\footnotesize\newlinetospace{\allcaps\plainauthor}}}%
    \fancyhead[RO]{\sffamily{\footnotesize\newlinetospace{\allcaps\plaintitle}}\footnotesize\quad \raisebox{1pt}{|}\quad{\footnotesize\thepage}}}
  {\fancyhead[RE,RO]{\smallcaps{\newlinetospace{\plaintitle}}\footnotesize\quad\thepage}}
  
%%%%%%%%%%%%%%%%%%%%%%%%%%%%%%%%%%%%%%%%%%%%%%%%%%%%%%%%%%%%
% Nomenclature package
\usepackage[intoc,refpage]{nomencl}
\renewcommand{\nomname}{Acronyms and List of Notations}
\def\pagedeclaration#1{, \hyperlink{page.#1}{page\nobreakspace#1}}
\makenomenclature

%%%%%%%%%%%%%%%%%%%%%%%%%%%%%%%%%%%%%%%%%%%%%%%%%%%%%%%%%%%%
% Bibliography Packages

\usepackage[natbib=true,%
backend=biber,%
style=alphabetic,%
sorting=anyvt,%
firstinits=true,%
doi=true,%
isbn=false,%
url=false,%
backref=true,%
texencoding=utf8,%
bibencoding=utf8,%
refsegment=chapter]{biblatex}
\IfFileExists{./Library.bib}{

 \addbibresource{./Library}
 
}{

  \addbibresource[location=remote]{https://gitlab.com/hsteinshiromoto/bibliography/raw/master/Library.bib}  

}



%\renewcommand*{\bibfont}{\footnotesize}

%%%%%%%%%%%%%%%%%%%%%%%%%%%%%%%%%%%%%%%%%%%%%%%%%%%%%%%%%%%%
% Definition of environments
\usepackage{thmtools}



\declaretheorem[style=plain,qed=$\lrcorner$,numberwithin=chapter]{theorem}
% Theorem environments

\declaretheorem[style=definition,qed=$\lrcorner$,sibling=theorem]{lemma}
\declaretheorem[style=definition,qed=$\lrcorner$,sibling=theorem]{corollary}
\declaretheorem[style=definition,qed=$\lrcorner$,sibling=theorem]{proposition}
\declaretheorem[style=definition,qed=$\lrcorner$,sibling=theorem]{claim}
\declaretheorem[style=definition,qed=$\lrcorner$,sibling=theorem]{conjecture}

% Remark and definition environments
\declaretheorem[style=definition,qed=$\lrcorner$,sibling=theorem]{assumption}
\declaretheorem[style=definition,qed=$\lrcorner$,sibling=theorem]{definition}
\declaretheorem[style=definition,qed=$\lrcorner$,sibling=theorem]{remark}
\declaretheorem[style=definition,qed=$\lrcorner$,sibling=theorem]{problem}

\declaretheorem[style=definition,qed=$\lrcorner$,numberwithin=chapter]{question}
\declaretheorem[style=definition,qed=$\lrcorner$,numbered=no]{answer}

% Example environments

\declaretheorem[style=definition,qed=$\lrcorner$,sibling=theorem]{example}

% Proof environment
\declaretheoremstyle[
  spaceabove=6pt, spacebelow=6pt,
  headfont=\normalfont\slshape,
  notefont=\normalfont\slshape, notebraces={(}{)},
  bodyfont=\normalfont,
  postheadspace=1em,
  qed=$\blacksquare$
]{proofstyle}
\declaretheorem[name=Proof,style=proofstyle,numbered=no]{theproof}

%%%%%%%%%%%%%%%%%%%%%%%%%%%%%%%%%%%%%%%%%%%%%%%%%%%%%%%%%%%%
% Custom commands

% Prints the month name (e.g., January) and the year (e.g., 2008)
\newcommand{\monthyear}{%
  \ifcase\month\or January\or February\or March\or April\or May\or June\or
  July\or August\or September\or October\or November\or
  December\fi\space\number\year
}

\usepackage{ragged2e} % To use the \justify command
% Prints an epigraph and speaker in sans serif, all-caps type.
\newcommand{\openepigraph}[2]{%
  %\sffamily\fontsize{14}{16}\selectfont
  \begin{fullwidth}
  \large
  \begin{doublespace}
  {\justify#1}\\% epigraph
  \noindent\hfill{\sffamily#2}% author
  \end{doublespace}
  \end{fullwidth}
}